%\documentclass[prd,twocolumn,amsmath,amssymb,floatfix,superscriptaddress,nofootinbib,preprintnumbers]{revtex4-1}
%\documentclass[prl,amsmath,amssymb,floatfix,superscriptaddress,nofootinbib,preprintnumbers]{revtex4-1}
\documentclass[prl,amsmath,amssymb,floatfix,superscriptaddress,nofootinbib,twocolumn]{revtex4-1}

\def\be{\begin{equation}}
\def\ee{\end{equation}}
\def\bea{\begin{eqnarray}}
\def\eea{\end{eqnarray}}
\newcommand{\vs}{\nonumber\\}
\def\across{a^\times}
\def\tcross{T^\times}
\def\ccross{C^\times}
\newcommand{\ec}[1]{Eq.~(\ref{eq:#1})}
\newcommand{\eec}[2]{Eqs.~(\ref{eq:#1}) and (\ref{eq:#2})}
\newcommand{\Ec}[1]{(\ref{eq:#1})}
\newcommand{\eql}[1]{\label{eq:#1}}
\newcommand{\sfig}[2]{
\includegraphics[width=#2]{../plots/#1}
        }
\newcommand{\sfigr}[2]{
\includegraphics[angle=270,origin=c,width=#2]{#1}
        }
\newcommand{\sfigra}[2]{
\includegraphics[angle=90,origin=c,width=#2]{#1}
        }
\newcommand{\Sfig}[2]{
   \begin{figure}[thbp]
   \begin{center}
    \sfig{#1.pdf}{\columnwidth}
    \caption{{\small #2}}
    \label{fig:#1}
     \end{center}
   \end{figure}
}
\newcommand{\Spng}[2]{
   \begin{figure}[thbp]
   \begin{center}
    \sfig{#1.png}{\columnwidth}
    \caption{{\small #2}}
    \label{fig:#1}
     \end{center}
   \end{figure}
}

\newcommand\dirac{\delta_D}
\newcommand{\rf}[1]{\ref{fig:#1}}
\newcommand\rhoc{\rho_{\rm cr}}
\newcommand\zs{D_S}
\newcommand\dts{\Delta t_{\rm Sh}}
\newcommand\zle{D_L}
\newcommand\zsl{D_{SL}}


%USEFUL PACKAGES
\usepackage[utf8]{inputenc}
\usepackage{graphicx}
\usepackage{amssymb}
\usepackage{amsmath}
\usepackage{bm}
\usepackage{color}
\usepackage{enumitem}
\usepackage[linktocpage=true]{hyperref} 
\hypersetup{
    colorlinks=true,       % false: boxed links; true: colored links
    linkcolor=red,          % color of internal links
    citecolor=blue,        % color of links to bibliography 
    filecolor=magenta,      % color of file links
    urlcolor=blue           % color of external links
}
\usepackage[all]{hypcap} 

%USEFUL MACROS
%\usepackage{myterms}
\definecolor{darkgreen}{cmyk}{0.85,0.1,1.00,0} 

\newcommand{\scott}[1]{{\color{darkgreen} #1}}
\newcommand{\peikai}[1]{{\color{blue} #1}}
\newcommand{\WH}[1]{{\color{red} #1}}
\newcommand{\wh}[1]{{\color{red} #1}}
\newcommand{\AL}[1]{{\color{magenta} AL: #1}}
\newcommand{\MR}[1]{{\color{blue} MR: #1}}

\newcommand{\nl}{\\ \indent}
%

\begin{document}

\title{Distortions in the Surface of Last Scattering}


\author{\large Peikai Li}
\author{\large Scott Dodelson}
\affiliation{Department of Physics, Carnegie Mellon University, Pittsburgh, Pennsylvania 15312, USA}
\author{\large Wayne Hu}
\affiliation{Kavli Institute for Cosmological Physics, Department of Astronomy \& Astrophysics,
Enrico Fermi Institute, The University of Chicago, Chicago, IL 60637, USA}
%\email{dodelson@fnal.gov}

\date{\today}

\begin{abstract}
The surface of last scattering of the photons in the cosmic microwave background is {\bf not} a spherical shell. Apart from its finite width, each photon experiences a different gravitational potential along its journey to us, leading to different travel times in different directions. Since all photons were released at the same cosmic time, the photons with longer travel times started farther away from us than those with shorter times. Thus, the surface of last scattering is corrugated, a deformed spherical shell. We present a quadratic estimator that could provide a map of the time delays as a function of position on the sky. The signal to noise of this map could exceed unity on large scales.
\end{abstract}

\maketitle

\section{Distance to the Last Scattering Surface}
\newcommand\fd{d}

The theory of general relativity dictates that particles traveling through gravitational potential wells experience time delays~\cite{1964PhRvL..13..789S}. If two photons are emitted at the same time, then they will travel different distances depending upon the potential $\Phi$ through which they travel. In the cosmological context of an expanding,  spatially flat background, the %fractional difference in comoving distance $D$ to a source at redshift $z$ is equal to  
%\be
%d(\hat n) = -2\, \int_0^{z} d\ln(1+z')\, K(z,z')\Phi\left(D(z')\hat n; t(z')\right)
%\ee
%\wh{see query on measure, I'd suggest also we switch back to: 
fractional difference in comoving distance
$D_*$ to a source at redshift $z_*$ is 
\be
d(\hat n) = \frac{2}{D_*}\, \int_0^{D_*} dD\, \Phi\left(D \hat n; t(D)\right)\eql{defd},
\ee
where %the kernel $K(z,z')=(H(z') D(z))^{-1}$ with $H$ the Hubble rate; and 
$t(D)$ is the age of the universe when the photon is a comoving distance $D$ from us, and  we use the space-time metric convention 
\bea
ds^{2}= -(1+2\Phi)dt^{2}+a^{2}(1-2\Phi)d\vec{x}^{2}
\eea
with $a(t)$ the scale factor.
Note the sign in \ec{defd}: if photons pass through an over-dense region where $\Phi<0$, then they experience a time delay and therefore they arrive from a closer distance than the unperturbed last scattering surface\footnote{There is also a geometric time delay that is typically of the same size for a single lens but is much smaller here on the large scales of interest.}. %, but when the path is through a series of peaks and troughs, the mean time delay is much smaller, of order $\theta_{\rm RMS}$, so we neglect it here.  }
%\wh{$\theta$ hasn't been defined and it's not clear how a time delay is being compared to an angle - in fact 
%actually I'm now unclear on what we're saying above -
%the geometric delay goes as deflection angles $\theta^2$ which don't cancel out whereas short wavelength potential delays do cancel out - so isn't the relevant comparison between
%the deflection angle and the angular scale of the delay, i.e. dipole scales of the full sky to arcmin scales of the lensing deflections?  In a single lens, multiple image system you'd set those two scales to be the same.  } 


Photons that comprise the cosmic microwave background (CMB) experience these same time delays or advances~\cite{Hu:2001yq} where $z_*$ is the redshift corresponding to the last scattering surface. Since photons do not decouple instantaneously from the electron-proton plasma, the surface of last scattering is often said to have a finite width, and a more accurate expression for the fractional difference in distance traveled is
%\be
%\fd(\hat n) = -2\, \int_0^{\infty} d\ln(1+z')\, e^{-\tau(z')}\, K(z,z')\Phi\left(D(z')\hat n; t(z')\right)
%%\frac{-2}{D_*} \int_0^\infty dD e^{-\tau(D)}\, \Phi\left(D\hat n; t(D)\right)
%\ee
%\wh{would suggest here 
\be
\fd(\hat n) = 2\, \int_0^{\infty} dz \, e^{-\tau(z)}\, K_\fd(z)\Phi\left(D(z)\hat n; t(z)\right)
\eql{deffield}
\ee
%\wh{perhaps define  $K_\fd(z) = -2(H(z)D_*)^{-1}$?}
where $H(z)$ is the Hubble expansion rate; $K_\fd(z) =  (H(z)D_*)^{-1}$ and $D_* =  \int_0^\infty dz' e^{-\tau(z') }/H(z')$.  Here 
 $\tau$ is the optical depth, ignoring reionization, which becomes very large at times smaller than the epoch of  last scattering, $t_*$ or equivalently when $z>z_*$.

This directional-dependent change in the distance to last scattering implies that the last scattering surface is not a simple spherical shell. There are two other well-studied phenomena that undercut the notion that the photons in the CMB freely streamed to us from a infinitely thin last scattering sphere. First, since the mean free path at recombination was finite, the last scattering surface has a finite width, and this is accounted for in all computations of CMB anisotropies. Second, the photons in the CMB experience angular deflections as they traverse the inhomogeneous universe~\cite{Hu:2001tn,Lewis:2006fu} and this effect has been exploited by recent experiments~\cite{Smith:2007rg,Ade:2013tyw,Story:2014hni,Sherwin:2016tyf,Aghanim:2018oex} that make maps of the projected gravitational potential.

Although deflections and delays are two different phenomena, they share some similarities, especially in the case of the CMB. Both are determined by the integrated potential along the line of sight, although with slightly different kernels, as depicted in Figure~\rf{kernel}: the integrated potential $\phi$ that determines deflections has the same form as the right-hand side of \ec{deffield}  with
%\scott{I changed the order of terms in the numerator here from what you had Wayne; I assume that was a typo? Please check out the new figure; it's different from before because of a bug; does it look right to you? is it misleading because we don't capture the transverse derivative for the deflection term?} \wh{I think it's now clear what is plotted - and its a good way to show high cross correlation though less good at identifying observability}
\be
K_\phi(z) = \frac{D_*-D(z)}{ D(z) D_* H(z)} .
\ee
%\wh{and define absolute instead of relative $K_\phi = (D_*-D)/(D D_* H)$.}
It is clear that they will be highly correlated, so as a first approximation, we might view the maps of the lensing potential created for example in \citet{Aghanim:2018oex} as maps of distance to the last scattering surface. Another similarity, one that has not yet been exploited, is that the quadratic estimator formalism  \cite{Hu:2001tn} can be applied to the delays as well, and this is what we will do in this paper.  We start though with the rather daunting facts that the RMS fractional distance differences are a factor of ten smaller than the RMS angular deviations 
 % this part is arguably comparing apples to oranges so I think the more daunting fact is the many orders of magnitude between their effects on the power spectra
 and their impact on CMB power spectra is even smaller \cite{Hu:2001yq}. Further, while the latter peaks at degree scales, the former peak on the largest scales where cosmic variance is
 higher.

\Sfig{kernel}{Kernel that weights the integral of the gravitational potential for the time delay examined here and the more carefully studied deflection angle.}

\section{Effect of distance changes on the CMB}
\newcommand\dtl{\delta\Theta^{\rm defl}}
\newcommand\dtd{\delta\Theta^{\rm dist}}
\newcommand\tob{\Theta^{\rm obs}}
\newcommand\tu{\tilde{\Theta}}
\newcommand\td{\Theta^{\rm dist}}
The %starting point is to assume that the last scattering surface is infinitely thin, note that the 
observed temperature in a given direction $\tob(\hat n)$ is the undistorted temperature $\tu(\hat n)$ plus the deflection due to gravitational lensing plus a term proportional to the small fractional difference $\fd(\hat n)$.  \wh{Linearizing these distortions, we obtain
\be
\tob(\hat n) = \tu(\hat n) - \nabla_{i} \phi(\hat n) \nabla^{i}\tu(\hat n) - \frac{\partial\tu(\hat n)}{\partial \ln D}\, \fd(\hat n),
\ee
where $D$ is the distance to the radiation sources, here mainly $D=D_*$ the distance 
to recombination.  We shall see below that we can express this radial derivative in
terms of operations on the radiation transfer function.
%\wh{I'm unclear on what caveat you are pointing out here - at this point, we're just linearizing and haven't yet
%dropped any cross terms as far as I can see: The first term is caused by lensing and the second term is due to time delay. 
%
%\scott{not sure where we're pointing out a caveat.}
%
% It was this sentence that I didn't understand: 
%Notice we don't consider their cross effect here. }
%
% so I think you are OK with commenting it out?
%
%The third term's derivative can be evaluated by recalling that the undistorted temperature can be expanded in spherical harmonics with coefficients
%\be
%\tu_{lm} = 4\pi(-i)^{l} \int \frac{d^{3}k}{(2\pi)^{3}}Y_{lm}^{*}(\hat{k})S(\vec k;\eta_{*})j_{l}(kD_{*}) ,
%\ee
%where $S$ is the source function, dominated on small scales by the monopole and dipole, as in Eq. (22) of Ref.~\cite{Hu:2001bc}. The derivative with respect to distance then acts on the spherical Bessel function, and after expanding the fractional difference in spherical harmonics as well, we obtain
%
% you seem to have defined conflicting macros and not changed this line:
%\be
%\tob_{lm} = \tu_{lm} -\delta \Theta_{lm}^{\rm lens}- \delta \td_{lm}
%\ee
% so: 
In harmonic space we can write
\be
\tob_{lm} = \tu_{lm} -\dtl_{lm}- \dtd_{lm}
\ee
with the two first order terms due to deflection and the change in distance equal to}
\bea
\dtl_{lm} &=& \sum_{LM}\sum_{l'm'}{}_{0}I_{lLl'}^{mMm'}\tu_{l'm'}\phi_{LM},\nonumber\\
\dtd_{lm} &=& \sum_{LM}\sum_{l'm'} {}_{0} J_{lLl'}^{mMm'}\frac{\partial \tu_{l'm'}}{\partial \ln D}d_{LM}\eql{dtd}
.\eea
Notice that both effects couple the undistorted temperature field to the observed
temperature field at a different multipole.
Here, we have written the integral over the product of three spherical harmonics as $_0I$ and $_0J$ to enable generalization to the case of polarization, which involves spin $s=2$ harmonics. The general expression is  
%\scott{I get two $(-1)^m$ factors: once when converting the complex conjugate and once when using Eq 12 from Okamoto-Hu.}\scott{Update: I think i now disagree with the $(-1)^{m_1}$ factor in Eq 12 of Okamoto-Hu, which means that I now agree with the $(-1)^m$'s here}
\bea
{}_{s}I_{lLl'}^{mMm'}  &=& (-1)^m\,
\bigl(\begin{smallmatrix} l & L & l' \\ -m & M & m'  \end{smallmatrix}\bigr) {}_{s}F_{lLl'}, \\
{}_{s}J_{lLl'}^{mMm'}  &=& (-1)^m\,
\bigl(\begin{smallmatrix} l & L & l' \\ -m & M & m'  \end{smallmatrix}\bigr) {}_{s}G_{lLl'}, \nonumber
\eea
%\scott{Resurrecting this comment of Wayne's to make sure we have it right, %although not sure i understand it:} \wh{why interchange the rolls of $L %\leftrightarrow l'$ for $d$? are you trying to get rid of some %$(-1)^{l+l'+L}$ phase }
%factor for $s\ne 0$?  relatedly $s$ looks unexplained here.}\scott{I think i addressed the last issue.}
%
% wh: OK, though I think permuting would affect s=2 and I didn't check that everything is consistent with this
% choice of phase factor
%
with
%\bea
%{}_{s}F_{lLl'} &\equiv& \big[L(L+1)-l'(l'+1)+l(l+1)\big]\nonumber  \\& \times& \frac{\big[ (2l+1)(2L+1)(2l'+1)\big]^{1/2} }{\sqrt{16 \pi}} \bigl(\begin{smallmatrix} l & L & l' \\ s & 0 & -s  \end{smallmatrix}\bigr) ,\nonumber\\
%{}_{s}G_{lLl'} &\equiv& \frac{\big[ (2l+1)(2L+1)(2l'+1)\big]^{1/2} }{\sqrt{4 \pi}} \bigl(\begin{smallmatrix} l & L & l' \\ s & 0 & -s  \end{smallmatrix}\bigr)\eql{fg}
%\eea
\wh{
\bea
{}_{s}F_{lLl'} &\equiv& \big[L(L+1)-l'(l'+1)+l(l+1)\big]  \\& \times&\sqrt{ \frac{ (2l+1)(2L+1)(2l'+1) }{16 \pi} }\begin{pmatrix} l & L & l'  \\   s & 0 & -s  \end{pmatrix} ,\nonumber\\
{}_{s}G_{lLl'} &\equiv& \sqrt{ \frac{(2l+1)(2L+1)(2l'+1) }{4 \pi}} \begin{pmatrix} l & L & l' \\  \vphantom{a^b} s & 0 & -s  \end{pmatrix}\eql{fg} \nonumber
.\eea
}
Note the extra two powers of the multipoles in the function $F$ that governs deflection; these follow from the fact that both the temperature and the potential are differentiated with respect to transverse position on the sky. By contrast, the radial derivative that governs the impact of the time delay, or change in distance to the last scattering surface, appears in \ec{dtd} as the logarithmic derivative of the undistorted coefficients $\tilde\Theta_{LM}$.

As in the case of the effect of deflections on the CMB, the varying distances to the last scattering surface leads to correlations between $l$-modes that differ from one another. 
\wh{First let us define the power spectrum of the undistorted fields
\begin{equation}
\tilde C_l^{\Theta\Theta} =  \frac{2}{\pi}\,\int_0^\infty k^2 dk P_{\cal R} T_l^ \Theta(k)T_l^ \Theta(k)
\eql{cdtt}
\end{equation}
where $T_\Theta[l;k]$ is the radiation transfer function and $P_{\cal R}$ is the  power spectrum of the initial curvature field ${\cal R}$.  Note that the transfer function is a radial integral over
the sources at a distance $D$, projected onto multipole moment $l$.
We proceed as in Ref.~\cite{Hu:2001tn} by focusing on the expectation of off-diagonal (${l_1,m_1\ne l_2,m_2}$) terms quadratic in the observed moments:
\bea
\langle \tob_{l_1m_1} \tob_{l_2m_2} \rangle &=& \sum_{LM}  (-1)^M \bigl(\begin{smallmatrix} l_1 & l_2 & L \\ m_1 & m_2 & -M  \end{smallmatrix}\bigr) \nonumber \\
%&\times&\left\{  \phi_{LM} \big[ \tilde{C}_{l_{1}}^{\Theta\Theta} {}_{0}F_{l_{2}Ll_{1}} +\tilde{C}_{l_{2}}^{\Theta\Theta} {}_{0}F_{l_{1}Ll_{2}}\big] \right. \nonumber \\
%&&+ \left. d_{LM} \big[ \tilde{C}_{l_{1}}^{\Theta\Theta,d} {}_{0}G_{l_{2}Ll_{1}} +\tilde{C}_{l_{2}}^{\Theta \Theta,d} {}_{0}G_{l_{1}Ll_{2}}\big] \right \} \nonumber \\
%&\equiv& \sum_{LM}(-1)^{M}\bigl(\begin{smallmatrix} l_1 & l_2 & L \\ m_1 & m_2 & -M  \end{smallmatrix}\bigr)
\vs&&\times\big[ \phi_{LM}f_{l_{1}Ll_{2}} + d_{LM}g_{l_{1}Ll_{2}} \big] ,  \eql{quad}
\eea
where
\bea
f_{l_1Ll_2} &\equiv& \big[ \tilde{C}_{l_{1}}^{\Theta\Theta} {}_{0}F_{l_{2}Ll_{1}} +\tilde{C}_{l_{2}}^{\Theta\Theta} {}_{0}F_{l_{1}Ll_{2}}\big],
\vs
g_{l_1Ll_2} &\equiv&\big[ \tilde{C}_{l_{1}}^{\Theta\Theta,d} {}_{0}G_{l_{2}Ll_{1}} +\tilde{C}_{l_{2}}^{\Theta \Theta,d} {}_{0}G_{l_{1}Ll_{2}}\big].\eql{lfg}
\eea
The change in distance to the last scattering spectrum produces the spectrum
\be
\tilde{C}^{\Theta\Theta,d}_{l} \equiv \frac{2}{\pi}\,\int_0^\infty  k^2 dk P_{\cal R} T_l^\Theta(k) 
T_l^{\Theta,d}(k).
\eql{cld}
\ee
This expression is identical to \ec{cdtt} other than the replacement of one of the transfer functions with 
\begin{equation}
T_l^{\Theta,d} \equiv \frac{\partial T_l^\Theta}{\partial \ln D},
\end{equation}
where the derivative is taken inside of the integrals over the radiation sources by modifying
the public CAMB code. 
The two spectra are shown in Fig.~\rf{TT1}. 
}

To clarify the meaning of these terms, consider the large scale limit where the temperature
source is the Sachs-Wolfe effect on the recombination surface at $D_*$, 
 $\Theta = {\cal R}/5$.   Then
 \begin{equation}
 T_l^\Theta(k) = \frac{1}{5} j_l(k D_*), \quad  T_l^{\Theta,d} (k) = \frac{1}{5}
 \frac{\partial j_l(k D_*)}{\partial \ln D_*}.
\end{equation}
More generally the modification to CAMB involves replacing the Bessel function kernel of
the source projection with its log derivative. 

%We modify the public CAMB code and account for the finite thickness of the last scattering surface, the actual integrand look like:
%\bea
%&&(S(k;\eta_*))^2\,j_l(kD_*)\,\frac{\partial j_l(kD_*)}{\partial\ln D_*} \vs
%\rightarrow \, && \mathcal{P}_{\rm R}(k)T_{\Theta}[j_l;k]T_{\Theta}^{(1)}[j_l;k] 
%\eea
%with $\mathcal{P}_{\rm R}$ primitive power spectrum, and $T_{\Theta}[j_l;k]$ is the transfer function for temperature defined as:
%\be
%T_{\Theta}[j_l;k] = \int d\eta \mathcal{S}_{\Theta}(k;\eta) j_l(kD(\eta))\eql{timeevol}
%.\ee
%which is the general form of the source function above. $\mathcal{S}_{\Theta}$ is the modified source function defined in CAMB, we can determine its expression using Eq. (22) of Ref.~\cite{Hu:2001bc} (ignore third term, which is the ISW effect),
%\bea
%S(\vec{k};\eta_{*})j_{l}(kD_{*}) &=& [\Theta+\Psi](\eta_{*})j_{l}(kD_{*})+v_{b}(k,\eta_{*})j'_{l}(kD_{*}) \vs
%\rightarrow T_{\Theta}[j_l;k] &=& \int d\eta \left\{ g(\eta) [\Theta+\Psi](\eta)\right. \vs
%&&\quad \quad\left.-(g(\eta)v_{b}(k,\eta))'\right\}j_{l}(kD(\eta))\eql{cld}
%\eea
%with $g(\eta)$ the visibility function, sharply peaked at $\eta=\eta_*$. Approximate $g(\eta)$ as $\delta(\eta-\eta_{*})$ we can go back from $T_{\Theta}[j_l;k]$ to $S(\vec{k};\eta_{*})j_{l}(kD_{*})$. And second term comes from integration by parts. Also define $T^{(1)}_{\Theta}[j_l;k] $ as $j_{l}(kD)$ in the integrand replaced by $\partial j_{l}(kD)/\partial\ln D$.
%In order to compute $C_{l}^{\Theta \Theta,d}$ we simply replace $j_{l}(kD)$ by $\partial j_{l}(kD)/\partial\ln D$ when calculating the transfer functions and we can get the result.

Note the difference between the two functions $\delta \Theta^{\rm defl}(\hat{n})$ and $\delta \Theta^{\rm dist}(\hat{n})$. Each involves a derivative. The one that governs deflections, $f$, %depends on the unlensed CMB spectrum $\tilde{C}_{l}^{\Theta\Theta}$
%and the deflections are encoded 
involves a derivative with respect to the transverse directions so $F$ as defined in \ec{fg} has more powers of $l$ than does $G$.  
% (which are derivatives with respect to transverse position in real space) contained in $F$ compared to $G$ in \ec{fg}. 
The function that governs changes in distances involved a radial derivative, and this shows up in the spectrum $\tilde{C}^{\Theta\Theta,d}_{l}$.

\Sfig{TT1}{Spectra of CMB temperature anisotropies and the logarithmic derivative of that spectrum with respect to the distance to the last scattering surface as defined in \ec{cld}.}


The correlation between different $l$-modes enables us, following Ref.~\cite{Okamoto:2003zw}, to extract information about the fields causing these correlations by forming quadratic estimators out of the observed temperature fields for both the gravitational potential responsible for deflections and the fractional distance field:
\bea
\hat \phi_{LM} &=& A_L 
%\frac{A_{L}}{\sqrt{L(L+1)}}
\sum_{l_1m_1}\sum_{l_2m_2} 
 (-1)^M  \bigl(\begin{smallmatrix} l_1 & l_2 & L \\ m_1 & m_2 & -M  \end{smallmatrix}\bigr) 
 \nonumber \\  && \times 
 h^{\phi}_{l_1l_2}(L)  \tob_{l_1m_1} \tob_{l_2m_2}\nonumber \\
\hat d_{LM} &=& B_{L} \sum_{l_1m_1}\sum_{l_2m_2}
(-1)^M  \bigl(\begin{smallmatrix} l_1 & l_2 & L \\ m_1 & m_2 & -M  \end{smallmatrix}\bigr) 
\nonumber \\ && \times
h^{d}_{l_1l_2}(L)  \tob_{l_1m_1} \tob_{l_2m_2}\eql{distest}
\eea
where
\bea
h^{\phi}_{l_1l_2}(L)&\equiv& \frac{f_{l_1Ll_2}}{2C_{l_1}C_{l_2}} \nonumber\\
h^{d}_{l_1l_2}(L)&\equiv& \frac{g_{l_1Ll_2}}{2C_{l_1}C_{l_2}}
\eea
and %\wh{redefined $A_L$ to absorb $L's$  - the original one was normalized to
%deflection angles but there's no reason to do that here.} %\peikai{I changed it back.}\scott{I changed the prefactor of $A_L$ (square root) so that now its expectation value is equal to $\phi$; please check.}
\bea
A_L &\equiv& 
%\sqrt{L(L+1)}
(2L+1)
 \left\{ \sum_{l_1l_2} h^{\phi}_{l_1l_2}(L)f_{l_1Ll_2}\right\}^{-1}\nonumber\\
B_L &\equiv& (2L+1) \left\{ \sum_{l_1l_2} h^{d}_{l_1l_2}(L)g_{l_1Ll_2}\right\}^{-1}.\eql{bt}
\eea 
With these definitions, the expectation values of the two estimators are equal to $\phi_{LM}$ and $d_{LM}$ respectively. \wh{comment on cross contamination of the estimators or say we
ignore it for now.
each are only unbiased in the absence of the other}
%\bea
%\langle \hat{\phi}_{LM} \rangle &=& \sqrt{L(L+1)} \phi_{LM} \\
%\langle \hat{d}_{LM} \rangle &=& d_{LM}
%\eea
%Notice we don't have the factor of $1/\sqrt{L(L+1)}$ for the time delay estimator, since time delay effect does not depend on angular gradient. This time delay estimator has an expected value equal to the true $d_{LM}$ and a variance 

The noise on these estimators is now given by the prefactors $A_L$ and $B_L$, so  
\be
\langle \hat d_{LM} \hat d^*_{L'M'}  \rangle = \delta_{LL'}\delta_{MM'} \left( C_L^{dd} + B_L \right)
\ee
with the first term on the right the signal and the second the noise. Fig.~\rf{Delay} shows the signal and noise at each $ L$ for several experimental configurations. Here, and throughout, the largest $l_{\rm max}$ we consider is 7000, as this seems to be within range being considered for a CMB-Stage 4 experiment (see Table 4.1 of Ref.~\cite{Abazajian:2016yjj}).

An estimate of the detectability of this signal can be obtained by computing the projected error, $\sigma_d$, on the amplitude $A^d$ of the power spectrum $A^dC_L^{dd}$, where the fiducial model has $A^d=1$.  Approximating the noise as Gaussian gives
\be
\left(\frac{1}{\sigma_d}\right)^2 = \sum_L^{\infty} \frac{(2L+1)f_{\rm sky}}{2} \left(\frac{C_L^{dd}}{C_L^{dd}+B_L}\right)^2\eql{sn}
.\ee
Fig.~\rf{Delay} shows that most of the signal comes from the lowest $L$-modes, particularly $L=1$. However, even for a full-sky experiment and the most optimistic noise projections, the
\wh{auto power spectrum $C_L^{dd}$ will not be measurable using temperature only.}%{\scott{Wayne: please check if this rewording is what you had in mind. Peikai: if we agree on this notation, please relabel the y-axis on figure 4 as ``Time Delay Detectability ($\sigma_d$)''}

\Sfig{Delay}{Signal (decreasing blue curve) due to the distorted surface of last scattering and the noise using the quadratic estimator constructed from the small scale temperature anisotropy ($\Theta\Theta$ in the notation of Table 1) for several different experimental configurations. Most optimistic is no noise out to $l_{\rm max}=7000$; the other two noise curves have sensitivity of 1$\mu$K-arcmin and beam size. %the and noise spectra for detector noise $C_{l}^{\Theta \Theta}|_{\rm noise} = 0$ or $\left(\frac{\Delta_{\Theta}}{T_{\rm CMB}} \right)^{2} e^{l(l+1)\theta_{\rm FWHM}^{2}/8 \, \rm ln \,2}$, with $\Delta_{\Theta}=1\mu K-arcmin$ and 
$\theta_{\rm FWHM}=1'$ or $4'$. Here, $f_{\rm sky}$ is set to one.} %\scott{Peikai, can you check fsky and lmax?} \wh{labels on figs need to be bigger, factors of $\ell$ need to be explained in text - for example if this were meant to read as averaged all $\ell,m$ in a log band $\Delta \ell/\ell \sim 1$ then the factor would be $1/\sqrt{\ell(2\ell+1)}$ and if it were to reflect errors then $\sqrt{2/\ell(2\ell+1)}$, etc..   }}

\section{Polarization}

The estimator above used only the temperature anisotropy field, but the polarization field contains even more information about the lensing potential that governs deflection and distance changes. This was worked out in detail by Ref.~\cite{Okamoto:2003zw} for deflection, and we follow their notation here. There are now three fields of interest: temperature $\Theta$, and the two fields associated with polarization, $E$ and $B$. With letters $a,b$ each ranging over these three fields, we have 
\bea
\langle a^{\rm obs}_{l_{1}m_{1}}b^{obs}_{l_{2}m_{2}}\rangle &=& \sum_{LM}(-1)^{M}\bigl(\begin{smallmatrix} l_1 & l_2 & L \\ m_1 & m_2 & -M  \end{smallmatrix}\bigr)\nonumber \\
&&\times
\big[ \phi_{LM}f^{\alpha}_{l_{1}Ll_{2}} + d_{LM}g^{\alpha}_{l_{1}Ll_{2}} \big] 
.\eea
The functions $f^\alpha$ and $g^\alpha$ are the generalizations of \ec{lfg} to include polarization (\ec{lfg} now corresponds to $\alpha=\Theta\Theta$). The full set of $f^{\alpha}$ was determined by Ref.~\cite{Okamoto:2003zw} and is reproduced in Table 1, which now includes the full set of $g^{\alpha}$ that govern the impact of changing radial distances. Note
that
\bea
\tilde{C}^{ab}_{l} \equiv \frac{2}{\pi}\,\int_0^\infty k^{2}dk \,&&{P}_{R}(k) T^{a}_l(k) T^b_l(k) \eea
denotes the power spectra of the undistorted fields with $T_a$ as the radiation transfer
function for the field $a$ and \be
\tilde{C}^{ab,d}_{l} \equiv  \frac{2}{\pi}\,\int_0^\infty k^{2}dk \,{P}_{R}(k)\frac{T_l^{a}(k)T^{b,d}(k)+T_l^{a,d}(k)T_l^{b}(k)}{2} 
%\left\{\right. \vs
%&& \left.+ T_l^{a,d}(k)T_l^{b}(k)\right\}
\ee
with 
\begin{equation}
T_l^{a,d} \equiv \frac{\partial T_l^a}{\partial \ln D}
\end{equation}
is again computed by modifying CAMB.  
 See Ref.~\cite{Hu:1997de} for a more detailed discussion.
Note that the angular deflection coefficients $f^\alpha$ do not carry superscript $^d$ because the derivatives are transverse and therefore captured by powers of $\ell$. %\wh{We still need to comment on the link to what is actually done, which is to include the time variation of the source fields}

\begin{table}[thbp]
\scalebox{0.9}{
\begin{tabular}{|l|c|c|}
\hline
$\alpha$ & $f_{l_{1}Ll_{2}}^{\alpha}$ & $g_{l_{1}Ll_{2}}^{\alpha}$  \\
\hline
$\Theta \Theta$ & $\tilde{C}_{l_{1}}^{\Theta\Theta} {}_{0}F_{l_{2}Ll_{1}}+\tilde{C}_{l_{2}}^{\Theta\Theta}{}_{0}F_{l_{1}Ll_{2}}$ & $\tilde{C}_{l_{1}}^{\Theta\Theta,d} {}_{0}G_{l_{2}Ll_{1}}+\tilde{C}_{l_{2}}^{\Theta\Theta,d}{}_{0}G_{l_{1}Ll_{2}}$\\
$\Theta E$ &$\tilde{C}_{l_{1}}^{\Theta E} {}_{2}F_{l_{2}Ll_{1}}+\tilde{C}_{l_{2}}^{\Theta E}{}_{0}F_{l_{1}Ll_{2}}$ & $\tilde{C}_{l_{1}}^{\Theta E,d} {}_{2}G_{l_{2}Ll_{1}}+\tilde{C}_{l_{2}}^{\Theta E,d}{}_{0}G_{l_{1}Ll_{2}}$ \\
$EE$ &$\tilde{C}_{l_{1}}^{EE} {}_{2}F_{l_{2}Ll_{1}}+\tilde{C}_{l_{2}}^{EE}{}_{2}F_{l_{1}Ll_{2}}$ & $\tilde{C}_{l_{1}}^{EE,d} {}_{2}G_{l_{2}Ll_{1}}+\tilde{C}_{l_{2}}^{EE,d}{}_{2}G_{l_{1}Ll_{2}}$ \\
$\Theta B$ & $i\tilde{C}^{ \Theta E}_{l_{1}} {}_{2}F_{l_{2}Ll_{1}}$ &$i\tilde{C}^{\Theta E,d}_{l_{1}}{}_{2}G_{l_{2}Ll_{1}}  $\\
$EB$ &$i\big[\tilde{C}_{l_{1}}^{EE} {}_{2}F_{l_{2}Ll_{1}}-\tilde{C}_{l_{2}}^{BB}{}_{2}F_{l_{1}Ll_{2}}$\big] & $i\big[\tilde{C}_{l_{1}}^{EE,d} {}_{2}G_{l_{2}Ll_{1}}-\tilde{C}_{l_{2}}^{BB,d}{}_{2}G_{l_{1}Ll_{2}}\big]$ \\
$BB$ &$\tilde{C}_{l_{1}}^{BB} {}_{2}F_{l_{2}Ll_{1}}+\tilde{C}_{l_{2}}^{BB}{}_{2}F_{l_{1}Ll_{2}}$ & $\tilde{C}_{l_{1}}^{BB,d} {}_{2}G_{l_{2}Ll_{1}}+\tilde{C}_{l_{2}}^{BB,d}{}_{2}G_{l_{1}Ll_{2}}$ \\
\hline
\end{tabular}}
\caption{Explicit forms for $f$ anb $h$ of various polarizations. Notice that for $TT$, $TE$, $EE$ and $BB$ polarization these functions are ``even"; for $TB$ and $EB$ polarization they are ``odd" instead. ``Even" and ``Odd" indicate that the functions are non-zero only when $l_{1}+l_{2}+L$ are even or odd, respectively.}
\end{table}

An estimator can now be constructed for each of the pairs of fields (except $BB$ \wh{what does this refer to given the table has BB - do you mean that we later assume the undistorted BB, $\tilde C_l^{BB}=0$}), so letting $\alpha$ denote pairs of fields $(ab)$, we have
\bea
\hat{d}^{\alpha}_{LM} &=&\nonumber  (-1)^{M} B_{L}^{\alpha}\sum_{l_{1}m_{1}}\sum_{l_{2}m_{2}}\bigl(\begin{smallmatrix} l_1 & l_2 & L \\ m_1 & m_2 & -M  \end{smallmatrix}\bigr) \\
&& \times h^{\alpha,d}_{l_{1}l_{2}}(L) a^{obs}_{l_{1}m_{1}}b^{obs}_{l_{2}m_{2}} ,
\eea
where the coefficients are generalizations of \ec{bt} with $g$ and $h$ there acquiring superscripts $\alpha$. The $g^\alpha$ are given in Table 1 and the minimizing weights are
\be
h^{\alpha=(ab),d}_{l_{1}l_{2}}(L) 
= \frac{C_{l_{2}}^{aa}C_{l_{1}}^{bb}g^{\alpha*}_{l_{1}Ll_{2}}-(-1)^{L+l_{1}+l_{2}}C_{l_{1}}^{ab}C_{l_{2}}^{ab}g^{\alpha*}_{l_{2}Ll_{1}}}{C_{l_{1}}^{aa}C_{l_{2}}^{aa}C_{l_{1}}^{bb}C_{l_{2}}^{bb}-(C_{l_{1}}^{ab}C_{l_{2}}^{ab})^{2}}
\ee
with 
\bea 
h_{l_{1}l_{2}}^{\alpha=(aa),d}(L)= \frac{g_{l_{1}Ll_{2}}^{\alpha*}}{2C_{l_{1}}^{aa}C_{l_{2}}^{aa}}.
\eea 
%We have some simplifications, for $a=b$, we have
%\be
%h^{\alpha,d}_{l_{1}l_{2}}(L) \rightarrow \frac{g^{\alpha *}_{l_{1}Ll_{2}}}{2C_{l_{1}}^{aa}C_{l_{2}}^{aa}}
%\ee
Note that in the special case when $C_{l}^{ab}=0$ (e.g., for $\Theta B$ or $EB$), this reduces to 
\be
h^{\alpha,d}_{l_{1}l_{2}}(L) \rightarrow \frac{g^{\alpha *}_{l_{1}Ll_{2}}}{C_{l_{1}}^{aa}C_{l_{2}}^{bb}}. 
\ee
The covariance of these quadratic estimators
\be
\langle \hat{d}^{\alpha*}_{LM}d^{\beta}_{L'M'}\rangle \equiv \delta_{LL'}\delta_{MM'}\big[ C_{L}^{dd}+N_{L}^{d,\alpha \beta} \big]
\ee
with Gaussian noise given by
\bea
N_{L}^{d,\alpha\beta}&=&\frac{B_{L}^{\alpha*}B_{L}^{\beta}}{2L+1}\sum_{l_{1}l_{2}}  \left\{ h_{l_{1}l_{2}}^{\alpha,d*} (L)\big[ C_{l_{1}}^{ac}C_{l_{2}}^{bd}h_{l_{1}l_{2}}^{\alpha,d}(L)\right. \nonumber \\
&&\left. +(-1)^{L+l_{1}+l_{2}}C_{l_{1}}^{ad}C_{l_{2}}^{bc} h_{l_{2}l_{1}}^{\beta,d}(L)  \big]\right\}\eql{full}
\eea
with $\alpha=(ab)$, $\beta=(cd)$. For $\alpha=\beta$, \ec{full} reduces to $N_{L}^{d,\alpha\alpha}=B_{L}^{\alpha}$.
Armed with these expressions, we can form a minimum variance estimator
\be
\hat{d}^{\rm mv}_{LM} = \sum_{\alpha}\omega^{\alpha}(L)\hat{d}^{\alpha}_{LM}
\ee
with weights and variance given by
\bea
\omega^{\alpha}(L) & =&N^{d,\rm mv}_{L} \sum_{\beta}(N_{L}^{d,-1})^{\alpha \beta} \\
N^{d,\rm mv}_{L} &=& \frac{1}{\sum_{\alpha\beta} (N_{L}^{d,-1})^{\alpha \beta}}
\eea
where $N_{L}^{d,-1}$ is the inverse matrix of time delay noise given by Eq. (27), with matrix indices given by polarizations. 

We saw in Fig.~\rf{Delay} that small scale temperature maps only are not sufficient to detect this signal convincingly. To assess the added information contained in the polarization field, we show the detectability in the form of $\sigma_d$ for the lowest $L$-modes ($L\leqslant 5$, which contributes all of the signal) as a function $l_{\rm max}$ for a noiseless experiment in Fig.~\rf{StoN}.
%We can see that the noise for time delay $\Theta \Theta$ quadratic estimator in FIG. 3 is much larger than the signal, given that we sum up from $1000$ to $4000$ for $l_{1}$ and $l_{2}$ in Eq. (19) indices and also detector noise is taken into consideration.

\Sfig{StoN}{Detectability of the spectrum of the time delay signal, $\sigma_d$, using \ec{sn} for different quadratic estimators as a function of the largest $l$-mode accessed. Here we set $f_{\rm sky}$ to be $1$.}
Here we keep $l_{\rm min}$ fixed at $1000$ for the CMB fields (we tested that our final results are insensitive to this choice) and let $l_{\rm max}$ vary.
We can see that at $l_{\rm max} \approx 5000$, the minimum variance estimator detectability reaches 1, \wh{and at 7000 is 1.X} but of course this is for the most optimistic of configurations. \wh{%What noise level is this?  if no noise why isn't $\Theta\Theta$ above one at $l_{\rm max}=7000$ as it seems to be in previous plot?  
Any comments on the differences between lensing and delay estimators especially on $B$ modes?} %$S/N \approx 1$.  \wh{some words on why $l_{\rm min}$? and whether that matters?}
According to Appendix B of Ref.~\cite{Hu:2000xe}, we can see that wigner-3j with odd $l_{1}+l_{2}+L$ is much smaller than its even counterpart. Unlike the deflection case, we don't have a compensation term in front. 
\wh{This is an obscure way of phrasing the explanation even to me. }
This would result in a much smaller detectability for estimators with B-modes, so we don't include those in Fig.~\rf{StoN}.


\section{Cross Power Spectrum}
The auto-spectrum of the distortion field, $d$, will apparently be very challenging to extract. Another possibility is to cross-correlate the quadratic estimator for the distortion field with other fields that are well-measured. Cross-correlations can be more easily detected if the two-fields are highly correlated and one of the fields can be detected with high signal to noise. 

As a first attempt, we consider the cross correlation of the two estimators without 
further optimization
\bea
 \langle \hat{\phi}^{\alpha*}_{LM}\hat{d}^{\beta}_{LM} \rangle= {C}_{L}^{\phi d} +N_{L}^{c},
\eea 
$c$ stands for cross here.  Note that the noise in the estimators are correlated because 
both come from the same quadratic combinations of observables
%where the bias is equal to the noise is
%where $\mathcal{N}_{L}^{\alpha \beta}$ is the bias of the expectation value of $\hat{\phi}^{\alpha*}_{LM}\hat{d}^{\beta}_{LM}$ and the power spectrum
%\bea
%\langle \hat{\phi}^{\alpha*}_{LM} \hat{d}^{\beta}_{L'M'}\rangle = \delta_{LL'}\delta_{MM'}(C_{L}^{\phi d}+\mathcal{N}_{L}^{\alpha \beta})
%\eea
%thus $\mathcal{N}^{\alpha \beta}_{L}$ is given similarly as $N_{L}^{\alpha \beta}$ by
\bea
N_{L}^{c}&=&\frac{A_{L}^{\alpha*}B_{L}^{\beta}}{(2L+1)}\sum_{l_{1}l_{2}}  \left\{ h_{l_{1}l_{2}}^{\alpha,\phi*} (L)\big[ C_{l_{1}}^{ac}C_{l_{2}}^{bd}h_{l_{1}l_{2}}^{\alpha,d}(L)\right. \nonumber \\
&&\left. +(-1)^{L+l_{1}+l_{2}}C_{l_{1}}^{ad}C_{l_{2}}^{bc} h_{l_{2}l_{1}}^{\beta,d}(L)  \big]\right\}.\eql{full} 
\eea
%Gaussian covariance of $\hat{C}_{LM}^{\phi d,\alpha \beta} $ is
%\bea 
%&&\big\langle \hat{C}_{LM}^{\phi d,\alpha \beta*} \hat{C}_{LM}^{\phi d,\gamma \sigma} - (C_{L}^{\phi d})^{2} \big\rangle/(2L+1) \nonumber \\
%&=&\left\{(C_{L}^{\phi d}+N_{L}^{c,\alpha \beta})(C_{L}^{\phi d}+N_{L}^{c,\gamma \sigma})\right. \nonumber \\
%&&+(C_{L}^{\phi d}+N_{L}^{c,\alpha \sigma})(C_{L}^{\phi d}+N_{L}^{c,\beta \gamma})\nonumber \\
%&&+(C_{L}^{\phi \phi}+N_{L}^{\phi,\alpha \gamma})(C_{L}^{d d}+N_{L}^{d,\beta \sigma})\nonumber\\
%&&-(C_{L}^{\phi d}+N_{L}^{c,\alpha \beta})N_{L}^{c,\gamma \sigma}-(C_{L}^{\phi d}+N_{L}^{c,\gamma \sigma})N_{L}^{c,\alpha \beta}\nonumber \\
%&&\left. +N_{L}^{c,\alpha \beta}N_{L}^{c,\gamma \sigma}-(C_{L}^{\phi d})^{2}\right\}/(2L+1)
%\eea
%When $(\alpha\beta)=(\gamma \sigma)$, this simplifies to
%\bea 
%&&\frac{1}{2L+1}\left\{(C_{L}^{\phi d}+N_{L}^{c,\alpha \beta})^{2}\right. \nonumber \\
%&&\left. 
%+(C_{L}^{\phi \phi} +N_{L}^{\phi,\alpha \alpha})(C_{L}^{d d}+N_{L}^{d,\beta \beta})\right\}
%\eea
%\noindent 
Assuming the estimator noise terms are Gaussian, we an  estimate the fractional
error on a measurement of the amplitude of the cross spectra $C_l^{\phi d}$ as
\bea
\eql{sn1}
%And we can compute cross power spectrum signal-to-noise by assuming that $(C_{L}^{\phi d}+\mathcal{N}_{L}^{\alpha \beta})=0$ and use minimu variance estimator for both lensing and time delay. Signal-to-noise in FIG.5 is given by:
\frac{1}{\sigma_{\rm cross}^{2} }&=& \sum_{L=1}^{L_{\rm max}}(2L+1)f_{\rm sky} \\
&&\times\frac{(C_L^{\phi d})^2}{( C_L^{\phi d}+N_{L}^{c})^2+
(C_L^{\phi\phi}+N_L^{\phi})(C_L^{dd}+N_L^{d})}. \nonumber
\eea
For the $\alpha\beta$ pair that corresponds to the minimum variance estimator of
$\phi$ and $d$ individually, we show this quantity in Fig.~\rf{CrossStoN} where we have set
 $f_{\rm sky}=1$. We  see that in this ideal case, $S/N$ could reach about $2.5$. 
 \Sfig{CrossStoN}{The detectability of the cross power spectrum of the distance distortion field and the potential inducing deflections as a function of $L_{\rm max}$. Unlike the auto-spectrum, there is signal out to $L_{\rm max}\sim 100$, but the contributions to the detectability plateau after that, so the best that can be hoped for with this cross spectrum is a $2.5\sigma$ detection
 for $l_{\rm max}=7000$.}

\section{Conclusions}

The last scattering surface of the CMB is not purely spherical due to the different travel times experienced by photons as they traverse the inhomogeneous gravitational potential. In principle, these distortions in the distance to different directions is detectable, but we conclude here that the standard auto-correlation techniques will not be sufficient to enable detection in the near future. There is the possibility of cross-correlating a map of the distance distortions constructed with the quadratic estimators introduced here with another map of a closely related integrated potential and extracting the signal in that way. Indeed, this was the way that the transverse distortions in the CMB were first detected~\cite{Smith:2007rg}. Here, we have considered  the cross-correlation signal between the distance distortion and the standard transverse deviation maps and concluded that even an all-sky experiment with superior angular resolution would be detect the cross-spectrum at only 3-sigma. We leave exploration of other cross-correlations for further work; however, we point out that yet another challenge confronted will be the fidelity of the distance distortion maps, as they are contaminated by the standard lensing signals. \wh{we need to add something to explain this  in the main text.  there used to be a small section on this
- restore some of that?}

%\begin{appendices}
%\appendix


%\section{Appendix B: Contamination}

%The presence of both the deflection and the distance perturbations to the CMB leads to contamination of each of the quadratic estimators, which implicitly neglected the other. For example, the distance estimator in \ec{distest} will get contributions from the deflection field $\delta\Theta^{\rm defl}$. Of course it is possible to construct estimators that are independent of one another, but for now, we simply estimate the magnitude of the contaminations.
%Consider a lensing quadratic estimator given by Okamoto \& Hu
%\bea
%\hat{\phi}^{\alpha}_{LM} &=&\nonumber \frac{A_{L}^{\alpha}}{\sqrt{L(L+1)}}\sum_{l_{1}m_{1}}\sum_{l_{2}m_{2}} \\
%&& (-1)^{M}\bigl(\begin{smallmatrix} l_1 & l_2 & L \\ m_1 & m_2 & -M  \end{smallmatrix}\bigr) h^{\alpha,\phi}_{l_{1}l_{2}}(L) a^{obs}_{l_{1}m_{1}}b^{obs}_{l_{2}m_{2}}
%\eea
%Here $h^{\alpha,\phi}_{l_{1}l_{2}}$ has a similar expression to $h^{\alpha,\phi}_{l_{1}l_{2}}$ with $g$ replaced by $f$. And we can compute the lensing contamination of this time delay estimator
%\bea
%&& \langle \hat{d}^{\alpha}_{LM} \rangle = d_{LM} +\phi_{LM}E_{L}^{\alpha}
%\eea
%where $E^{\alpha}_{L}$ is given by
%\be
%E^{\alpha}_{L} = \frac{\sum_{l_{1}l_{2}}h_{l_{1}l_{2}}^{\alpha,d}(L)f_{l_{1}Ll_{2}}^{\alpha}}{\sum_{l_{1}l_{2}}h_{l_{1}l_{2}}^{\alpha,d}(L)g_{l_{1}Ll_{2}}^{\alpha}}
%\ee
%And contamination from lensing is relatively large, $E_{L=1}^{TT} \approx -1.73$. We would like to see if it is possible to eliminate this effect by constructing a new time delay estimator
%\bea
%\hat{\mathcal{D}}_{L}^{\alpha} = %\hat{d}_{LM}^{\alpha}+\delta\hat{d}_{LM}^{\alpha}
%\eea 
%where $\delta\hat{d}_{LM}^{\alpha}$ satisfies property
%\bea 
%\langle \delta\hat{d}_{LM}^{\alpha} \rangle = - \phi_{LM}E_{L}^{\alpha}
%\eea 
%so that
%\bea 
%\langle \hat{\mathcal{D}}_{LM}^{\alpha} \rangle = d_{LM}
%\eea 
%We should also notice that we only consider $\alpha = TT, TE, EE$ cases. \\
%Let's see what the Gaussion noise would be in this construction
%\bea 
%&&\langle \hat{\mathcal{D}}_{LM}^{\alpha}\hat{\mathcal{D}}_{L'M'}^{\beta*} \rangle \nonumber \\
%&=&\langle \hat{d}_{LM}^{\alpha}\hat{d}_{L'M'}^{\beta*} \rangle + \langle \delta\hat{d}_{LM}^{\alpha}\hat{d}_{L'M'}^{\beta*} \rangle \nonumber \\
%&&+\langle \hat{d}_{LM}^{\alpha}\delta\hat{d}_{L'M'}^{\beta*} \rangle + \langle \delta\hat{d}_{LM}^{\alpha}\delta\hat{d}_{L'M'}^{\beta*} \rangle \nonumber \\
%&=&\delta_{LL'}\delta_{MM'}(C_{L}^{dd}+N_{L}^{\alpha\beta}+\Delta_{L}^{\alpha\beta})
%\eea 
%\end{appendices}

\bibliography{refs}
\end{document}

 \subsection{Comparison with Deflection estimator}
 
 It is interesting to compare this estimator with that obtained for the deflection angle at this stage in the calculation in \citet{Okamoto:2003zw}. Paraphrasing Eq (29) there leads to
 \be
 \hat\alpha_{LM} = \frac{A_L}{\sqrt{L(L+1}}\,\sum_{l_1,m_1,l_2m_2}
 (-1)^M \left(\begin{matrix} l_1& l_2 &L\cr m_1&m_2&-M \end{matrix}\right)
 g_{l_1l_2}
 a^{\rm obs}_{l_1m_1} a^{\rm obs}_{l_2m_2} 
 \ee
 where
 
 \be
 A_L \equiv L(L+1)(2L+1) \left[ \sum_{l_1l_2} g_{l_1l_2} f_{l_1Ll_2}\right]^{-1}
 \ee
