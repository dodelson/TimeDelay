%\documentclass[prd,twocolumn,amsmath,amssymb,floatfix,superscriptaddress,nofootinbib,preprintnumbers]{revtex4-1}
%\documentclass[prl,amsmath,amssymb,floatfix,superscriptaddress,nofootinbib,preprintnumbers]{revtex4-1}
\documentclass[prl,amsmath,amssymb,floatfix,superscriptaddress,nofootinbib,twocolumn]{revtex4-1}

\def\be{\begin{equation}}
\def\ee{\end{equation}}
\def\bea{\begin{eqnarray}}
\def\eea{\end{eqnarray}}
\newcommand{\vs}{\nonumber\\}
\def\across{a^\times}
\def\tcross{T^\times}
\def\ccross{C^\times}
\newcommand{\ec}[1]{Eq.~(\ref{eq:#1})}
\newcommand{\eec}[2]{Eqs.~(\ref{eq:#1}) and (\ref{eq:#2})}
\newcommand{\Ec}[1]{(\ref{eq:#1})}
\newcommand{\eql}[1]{\label{eq:#1}}
\newcommand{\sfig}[2]{
\includegraphics[width=#2]{../plots/#1}
        }
\newcommand{\sfigr}[2]{
\includegraphics[angle=270,origin=c,width=#2]{#1}
        }
\newcommand{\sfigra}[2]{
\includegraphics[angle=90,origin=c,width=#2]{#1}
        }
\newcommand{\Sfig}[2]{
   \begin{figure}[thbp]
   \begin{center}
    \sfig{#1.pdf}{0.9\columnwidth}
    \caption{{\small #2}}
    \label{fig:#1}
     \end{center}
   \end{figure}
}
\newcommand{\Spng}[2]{
   \begin{figure}[thbp]
   \begin{center}
    \sfig{#1.png}{\columnwidth}
    \caption{{\small #2}}
    \label{fig:#1}
     \end{center}
   \end{figure}
}

\newcommand\dirac{\delta_D}
\newcommand{\rf}[1]{\ref{fig:#1}}
\newcommand\rhoc{\rho_{\rm cr}}
\newcommand\zs{D_S}
\newcommand\dts{\Delta t_{\rm Sh}}
\newcommand\zle{D_L}
\newcommand\zsl{D_{SL}}


%USEFUL PACKAGES
\usepackage[utf8]{inputenc}
\usepackage{graphicx}
\usepackage{amssymb}
\usepackage{amsmath}
\usepackage{bm}
\usepackage{color}
\usepackage{enumitem}
\usepackage[linktocpage=true]{hyperref} 
\hypersetup{
    colorlinks=true,       % false: boxed links; true: colored links
    linkcolor=red,          % color of internal links
    citecolor=blue,        % color of links to bibliography 
    filecolor=magenta,      % color of file links
    urlcolor=blue           % color of external links
}
\usepackage[all]{hypcap} 

%USEFUL MACROS
%\usepackage{myterms}
\definecolor{darkgreen}{cmyk}{0.85,0.1,1.00,0} 

\newcommand{\SD}[1]{{\color{darkgreen} SD: #1}}
\newcommand{\WH}[1]{{\color{red} WH: #1}}
\newcommand{\wh}[1]{{\color{red} WH: #1}}
\newcommand{\AL}[1]{{\color{magenta} AL: #1}}
\newcommand{\MR}[1]{{\color{blue} MR: #1}}

\newcommand{\nl}{\\ \indent}
%

\begin{document}

\title{Distortions in the Surface of Last Scattering}


\author{\large Peikai Li}
\author{\large Scott Dodelson}
%\email{dodelson@fnal.gov}
\affiliation{Department of Physics, Carnegie Mellon University, Pittsburgh, Pennsylvania 15312, USA.}

\date{\today}

\begin{abstract}
The surface of last scattering of the photons in the cosmic microwave background is {\bf not} a spherical shell. Apart from its finite width, each photon experiences a difference gravitational potential along its journey to us, leading to different travel times in different directions. Since all photons were released at the same cosmic time, the photons with longer travel times started farther away from us than those with shorter times. Thus, the surface of last scattering is corrugated, a deformed spherical shell. We present a quadratic estimator that could provide a map of the time delays as a function of position on the sky. The signal to noise of this map could exceed unity on large scales.
\end{abstract}

\maketitle

\section{Distance to the Last Scattering Surface}

The theory of general relativity dictates that particles traveling through gravitational potential wells experience time delays~\cite{1964PhRvL..13..789S}. If two photons are emitted at the same time, then they will travel different distances depending upon the potential $\Phi$ through which they travel. In a cosmological context of an expanding universe, the difference in comoving distance $D$ is equal to  
\be
d(\hat n) = \frac{-2}{D} \int_0^D dD'\, \Phi\left(D'\hat n; t(D')\right)
\ee
where $t(D)$ is the age of the universe at the time that the photon is a distance $D$ from us. There is also a geometric time delay that is typically of the same size for a single lens, but when the path is through a series of peaks and troughs, the mean time delay is much smaller, of order $\theta_{RMS}$, so we neglect it here. 

Similarly~\cite{Hu:2001yq}, photons that comprise the cosmic microwave background (CMB) experience these same time delays or advances depending on the integrated potential through which they travel. Since photons do not decouple instantaneously from the electron-proton plasma, the surface of last scattering is often said to have a finite width, and a more accurate expression for the fractional difference in distance traveled is
\be
d^{\rm CMB}(\hat n) = \frac{-2}{D_*} \int_0^\infty dD e^{-\tau(D)}\, \Phi\left(D\hat n; t(D)\right)
\ee
where $\tau$ is the optical depth, which becomes very large at times smaller than $t_*$ or equivalently when $D>D_*$, the nominal distance to the last scattering surface. This directional-dependent change in the distance to last scattering is independent of its finite width and a phenomenon different than the angular deflections~\cite{Hu:2001tn,Lewis:2006fu} that have been captured by recent experiments~\cite{Smith:2007rg,Ade:2013tyw,Story:2014hni,Sherwin:2016tyf,Aghanim:2018oex}.

Although deflections and delays are two different phenomena, they share some similarities, especially in the case of the CMB. Both are determined by the integrated potential along the line of sight, although with slight different kernels, as depicted in Figure~\rf{kernel}. It is clear that they will be highly correlated, so as a first approximation, we might view the maps of the lensing potential created for example in \citet{Aghanim:2018oex} as maps of distance to the last scattering surface. Another similarity, one that has not yet been exploited, is that the formalism first proposed in \citet{Hu:2001tn} can be applied to the delays as well, and this is what we will do in this paper. We start though with the rather humbling calculation of \cite{Hu:2001yq} that the RMS fractional distance differences are a factor of ten smaller than the RMS angular deviations. Further, while the latter peaks at degree scales, the former peak on the largest scales.

\Sfig{kernel}{Kernel that weights the integral of the gravitational potential for the time delay examined here and the more carefully studied deflection angle.}

\section{Effect of distance changes on the CMB}
\newcommand\fd{d^{\rm CMB}}
\newcommand\tob{\Theta^{\rm obs}}
\newcommand\tu{\tilde{\Theta}}
\newcommand\td{\Theta^{\rm dist}}
A starting point is to assume that the last scattering surface is infinitely thin, note that the observed temperature in a given direction $\tob(\hat n)$ is the undistorted temperature $\tu(\hat n)$ plus a term proportional to the small fractional difference $\fd(\hat n)$:
\be
\tob(\hat n) = \tu(\hat n) + \nabla_{i} \phi(\hat n) \nabla^{i}\tu(\hat n) + \frac{\partial\tu(\hat n)}{\partial \ln D_*}\, \fd(\hat n)
.\ee
The first term is caused by lensing and has been fully evaluated in Okamoto \& Hu. The second term is due to time delay. Notice we don't consider their cross effect here. The second term's derivative can be evaluated by recalling that the undistorted temperature can be expanded in spherical harmonics with coefficients
\be
\tu_{lm} = 4\pi(-i)^{l} \int \frac{d^{3}k}{(2\pi)^{3}}Y_{lm}^{*}(\hat{k})S(\vec k;\eta_{*})j_{l}(kD_{*}) 
\ee
where $S$ is the source function, dominated on small scales by the monopole and dipole, as in Eq. (22) of Ref.~\cite{Hu:2001bc} (but note. The derivative then acts on the spherical Bessel function, and after expanding the fractional difference in spherical harmonics as well, we obtain
\be
\tob_{lm} = \tu_{lm} +\delta \Theta_{lm}^{\rm lens}+ \delta \td_{lm}
\ee
with the first order term due to the distortion equal to
\bea
\delta \Theta_{lm}^{\rm lens} &=& \sum_{LM}\sum_{l'm'}\tu_{l'm'} \;{}_{0}I_{lLl'}^{mMm'}\phi_{LM}\\
\delta \Theta_{lm}^{\rm dist} &=& \sum_{LM}\sum_{l'm'} \frac{\partial \tu_{LM}(D_{*})}{\partial \ln D_{*}} \;{}_{0} J_{lLl'}^{mMm'}d_{l'm'}
.\eea
Here, the integral over the product of three spherical harmonics is expressed as (general case)
\bea
{}_{s}I_{lLl'}^{mMm'}  &=& (-1)^m\,
\bigl(\begin{smallmatrix} l & L & l' \\ -m & M & m'  \end{smallmatrix}\bigr) {}_{s}F_{lLl'} \\
{}_{s}J_{lLl'}^{mMm'}  &=& (-1)^m\,
\bigl(\begin{smallmatrix} l & L & l' \\ -m & M & m'  \end{smallmatrix}\bigr) {}_{s}G_{lLl'}
\eea
with
\bea
{}_{s}F_{lLl'} &\equiv& \big[L(L+1)-l'(l'+1)+l(l+1)\big]\nonumber  \\& \times& \frac{\big[ (2l+1)(2L+1)(2l'+1)\big]^{1/2} }{\sqrt{16 \pi}} \bigl(\begin{smallmatrix} l & L & l' \\ s & 0 & -s  \end{smallmatrix}\bigr) \\
{}_{s}G_{lLl'} &\equiv& \frac{\big[ (2l+1)(2L+1)(2l'+1)\big]^{1/2} }{\sqrt{4 \pi}} \bigl(\begin{smallmatrix} l & L & l' \\ s & 0 & -s  \end{smallmatrix}\bigr)
.\eea

\Spng{TT1}{Spectra of CMB temperature anisotropies and the logarithmic derivative of that spectrum with respect to the distance to the last scattering surface as defined in \ec{cld}.}
We proceed as in Ref.~\cite{Hu:2001tn} by focusing on the expectation of off-diagonal (${l_1,m_1\ne l_2,m_2}$) terms quadratic in the observed moments:
\bea
&&\langle \tob_{l_1m_1} \tob_{l_2m_2} \rangle = \sum_{LM}  (-1)^M \bigl(\begin{smallmatrix} l_1 & l_2 & L \\ m_1 & m_2 & -M  \end{smallmatrix}\bigr) \nonumber \\
&\times&\left\{  \phi_{LM} \big[ \tilde{C}_{l_{1}}^{\Theta\Theta} {}_{0}F_{l_{2}Ll_{1}} +\tilde{C}_{l_{2}}^{\Theta\Theta} {}_{0}F_{l_{1}Ll_{2}}\big] \right. \nonumber \\
&&+ \left. d_{LM} \big[ \tilde{C}_{l_{1}}^{\Theta\Theta,d} {}_{0}G_{l_{2}Ll_{1}} +\tilde{C}_{l_{2}}^{\Theta \Theta,d} {}_{0}G_{l_{1}Ll_{2}}\big] \right \} \nonumber \\
&=& \sum_{LM}(-1)^{M}\bigl(\begin{smallmatrix} l_1 & l_2 & L \\ m_1 & m_2 & -M  \end{smallmatrix}\bigr)\nonumber \\
&\times&\big[ \phi_{LM}f_{l_{1}Ll_{2}} + d_{LM}g_{l_{1}Ll_{2}} \big] 
\eea
where the spectrum
\be
C^{\Theta\Theta,d}_{l} \equiv \frac{2}{\pi}\,\int_0^\infty dk\,k^2\, P_S(k,\eta_*)\,j_l(kD_*)\,\frac{\partial j_l(kD_*)}{\partial\ln D_*}\eql{cld}
\ee
with $P_S$ the power spectrum of the source. This expression is identical to the one for the undistorted CMB spectrum $C_l$ other than the replacement $j_l\rightarrow j_l'$. The two spectra are shown in Fig.~\rf{TT1}. We modify the public CAMB code and get the result. Therefore, following Ref.~\cite{Okamoto:2003zw}, we form the quadratic estimator for gravitational lensing and also the fractional distance field
\bea
\hat \phi_{LM} &=& \frac{A_{L}}{\sqrt{L(L+1)}}\sum_{l_1m_1}\sum_{l_2m_2} \nonumber \\  
&& (-1)^M  \bigl(\begin{smallmatrix} l_1 & l_2 & L \\ m_1 & m_2 & -M  \end{smallmatrix}\bigr) h^{\phi}_{l_1l_2}(L)  \tob_{l_1m_1} \tob_{l_2m_2} \\
\hat d_{LM} &=& B_{L} \sum_{l_1m_1}\sum_{l_2m_2}\nonumber \\
&& (-1)^M  \bigl(\begin{smallmatrix} l_1 & l_2 & L \\ m_1 & m_2 & -M  \end{smallmatrix}\bigr) h^{d}_{l_1l_2}(L)  \tob_{l_1m_1} \tob_{l_2m_2}
\eea
where
\bea
h^{\phi}_{l_1l_2}(L)&\equiv& \frac{f_{l_1Ll_2}}{2C_{l_1}C_{l_2}} \\
h^{d}_{l_1l_2}(L)&\equiv& \frac{h_{l_1Ll_2}}{2C_{l_1}C_{l_2}}
\eea
and
\bea
A_L &\equiv& L(L+1)(2L+1) \left\{ \sum_{l_1l_2} h^{\phi}_{l_1l_2}(L)f_{l_1Ll_2}\right\}^{-1}\\
B_L &\equiv& (2L+1) \left\{ \sum_{l_1l_2} h^{d}_{l_1l_2}(L)g_{l_1Ll_2}\right\}^{-1}
\eea 
Notice we don't have the factor of $1/\sqrt{L(L+1)}$ for the time delay estimator, since time delay effect does not depend on angular gradient. This time delay estimator has an expected value equal to the true $d_{LM}$ and a variance 
\be
\langle \hat d_{LM} \hat d^*_{L'M'}  \rangle = \delta_{LL'}\delta_{MM'} \left( C_L^{dd} + B_L \right)
\ee
with the first term on the right the signal and the second the noise. Fig.~\rf{Delay} shows the signal and noise as a function of $L$ for a specific choice of experimental configuration.
\Spng{Delay}{Signal and noise spectra for detector noise $C_{l}^{\Theta \Theta}|_{\rm noise} = \left(\frac{\Delta_{\Theta}}{T_{\rm CMB}} \right)^{2} e^{l(l+1)\theta_{\rm FWHM}^{2}/8 \, \rm ln \,2}$, \\with $\Delta_{\Theta}=1\mu K-arcmin$ and $\theta_{\rm FWHM}=4'$.}

\section{Polarization and Minimum Variance Estimator}
Based on Okamoto \& Hu's construction, we can make a similar table for function $f$ and $g$ in the following expression
\bea
\langle a^{obs}_{l_{1}m_{1}}b^{obs}_{l_{2}m_{2}}\rangle &=& \sum_{LM}(-1)^{M}\bigl(\begin{smallmatrix} l_1 & l_2 & L \\ m_1 & m_2 & -M  \end{smallmatrix}\bigr)\nonumber \\
&\times&\big[ \phi_{LM}f^{\alpha}_{l_{1}Ll_{2}} + d_{LM}g^{\alpha}_{l_{1}Ll_{2}} \big] 
\eea
where $\alpha=ab$ stands for different polarizations.

\begin{table}[thbp]
\scalebox{0.9}{
\begin{tabular}{|l|c|c|}
\hline
$\alpha$ & $f_{l_{1}Ll_{2}}^{\alpha}$ & $g_{l_{1}Ll_{2}}^{\alpha}$  \\
\hline
$\Theta \Theta$ & $\tilde{C}_{l_{1}}^{\Theta\Theta} {}_{0}F_{l_{2}Ll_{1}}+\tilde{C}_{l_{2}}^{\Theta\Theta}{}_{0}F_{l_{1}Ll_{2}}$ & $\tilde{C}_{l_{1}}^{\Theta\Theta,d} {}_{0}G_{l_{2}Ll_{1}}+\tilde{C}_{l_{2}}^{\Theta\Theta,d}{}_{0}G_{l_{1}Ll_{2}}$\\
$\Theta E$ &$\tilde{C}_{l_{1}}^{\Theta E} {}_{2}F_{l_{2}Ll_{1}}+\tilde{C}_{l_{2}}^{\Theta E}{}_{0}F_{l_{1}Ll_{2}}$ & $\tilde{C}_{l_{1}}^{\Theta E,d} {}_{2}G_{l_{2}Ll_{1}}+\tilde{C}_{l_{2}}^{\Theta E,d}{}_{0}G_{l_{1}Ll_{2}}$ \\
$EE$ &$\tilde{C}_{l_{1}}^{EE} {}_{2}F_{l_{2}Ll_{1}}+\tilde{C}_{l_{2}}^{EE}{}_{2}F_{l_{1}Ll_{2}}$ & $\tilde{C}_{l_{1}}^{EE,d} {}_{2}G_{l_{2}Ll_{1}}+\tilde{C}_{l_{2}}^{EE,d}{}_{2}G_{l_{1}Ll_{2}}$ \\
$\Theta B$ & $i\tilde{C}^{ \Theta E}_{l_{1}} {}_{2}F_{l_{2}Ll_{1}}$ &$i\tilde{C}^{\Theta E,d}_{l_{1}}{}_{2}G_{l_{2}Ll_{1}}  $\\
$EB$ &i\big[$\tilde{C}_{l_{1}}^{EE} {}_{2}F_{l_{2}Ll_{1}}-\tilde{C}_{l_{2}}^{BB}{}_{2}F_{l_{1}Ll_{2}}$\big] & $i\big[\tilde{C}_{l_{1}}^{EE,d} {}_{2}G_{l_{2}Ll_{1}}-\tilde{C}_{l_{2}}^{BB,d}{}_{2}G_{l_{1}Ll_{2}}\big]$ \\
$BB$ &$\tilde{C}_{l_{1}}^{BB} {}_{2}F_{l_{2}Ll_{1}}+\tilde{C}_{l_{2}}^{BB}{}_{2}F_{l_{1}Ll_{2}}$ & $\tilde{C}_{l_{1}}^{BB,d} {}_{2}G_{l_{2}Ll_{1}}+\tilde{C}_{l_{2}}^{BB,d}{}_{2}G_{l_{1}Ll_{2}}$ \\
\hline
\end{tabular}}
\caption{Explicit forms for $f$ anb $h$ of various polarizations. Notice that for $TT$, $TE$, $EE$ and $BB$ polarization these functions are "even"; for $TB$ and $EB$ polarization they are "odd" instead. "Even" and "Odd" indicate that the functions are non-zero only when $l_{1}+l_{2}+L$ are even or odd, respectively.}
\end{table}
with corresponding estimators given by:
\bea
\hat{d}^{\alpha}_{LM} &=&\nonumber B_{L}^{\alpha}\sum_{l_{1}m_{1}}\sum_{l_{2}m_{2}} \\
&& (-1)^{M}\bigl(\begin{smallmatrix} l_1 & l_2 & L \\ m_1 & m_2 & -M  \end{smallmatrix}\bigr) h^{\alpha,d}_{l_{1}l_{2}}(L) a^{obs}_{l_{1}m_{1}}b^{obs}_{l_{2}m_{2}}
\eea
Also notice that we only consider off-diagonal terms as well.
Here the minimum weighting function is given by
\bea
 && \noindent h^{\alpha,d}_{l_{1}l_{2}}(L) \nonumber \\
&=& \frac{C_{l_{2}}^{aa}C_{l_{1}}^{bb}g^{\alpha*}_{l_{1}Ll_{2}}-(-1)^{L+l_{1}+l_{2}}C_{l_{1}}^{ab}C_{l_{2}}^{ab}g^{\alpha*}_{l_{2}Ll_{1}}}{C_{l_{1}}^{aa}C_{l_{2}}^{aa}C_{l_{1}}^{bb}C_{l_{2}}^{bb}-(C_{l_{1}}^{ab}C_{l_{2}}^{ab})^{2}}
\eea
We have some simplifications, for $a=b$, we have
\be
h^{\alpha,d}_{l_{1}l_{2}}(L) \rightarrow \frac{g^{\alpha *}_{l_{1}Ll_{2}}}{2C_{l_{1}}^{aa}C_{l_{2}}^{aa}}
\ee
and for $C_{l}^{ab}=0$ (e.g., for $\Theta B$ or $EB$),
\be
h^{\alpha,d}_{l_{1}l_{2}}(L) \rightarrow \frac{g^{\alpha *}_{l_{1}Ll_{2}}}{C_{l_{1}}^{aa}C_{l_{2}}^{aa}}
\ee
The covariance of these quadratic estimator
\be
\langle \hat{d}^{\alpha*}_{LM}d^{\beta}_{L'M'}\rangle \equiv \delta_{LL'}\delta_{MM'}\big[ C_{L}^{dd}+N_{L}^{\alpha \beta} \big]
\ee
with Gaussion noise given by
\bea
N_{L}^{\alpha\beta}&=&\frac{B_{L}^{\alpha*}B_{L}^{\beta}}{2L+1}\sum_{l_{1}l_{2}}  \left\{ h_{l_{1}l_{2}}^{\alpha,d*} (L)\big[ C_{l_{1}}^{ac}C_{l_{2}}^{bd}h_{l_{1}l_{2}}^{\alpha,d}(L)\right. \nonumber \\
&&\left. +(-1)^{L+l_{1}+l_{2}}C_{l_{1}}^{ad}C_{l_{2}}^{bc} h_{l_{2}l_{1}}^{\beta,d}(L)  \big]\right\}
\eea
with $\alpha=(ab)$, $\beta=(cd)$. For $\alpha=\beta$, simply we have $N_{L}^{\alpha\alpha}=B_{L}^{\alpha}$.
We can thus form a minimum variance estimator
\be
\hat{d}^{\rm mv}_{LM} = \sum_{\alpha}\omega^{\alpha}(L)\hat{d}^{\alpha}_{LM}
\ee
with weights and variance give by
\bea
\omega^{\alpha}(L) & =&A^{\rm mv}_{L} \sum_{\beta}(N_{L}^{-1})^{\alpha \beta} \\
N^{\rm mv}_{L} &=& \frac{1}{\sum_{\alpha\beta}} (N_{L}^{-1})^{\alpha \beta}
\eea
where $N_{L}^{-1}$ is the inverse matrix of time delay noise given by Eq. (27), with matrix indices given by polarizations. 

We can see that the noise for time delay $\Theta \Theta$ quadratic estimator in FIG. 3 is much larger than the signal, given that we sum up from $1000$ to $4000$ for $l_{1}$ and $l_{2}$ in Eq. (19) indices and also detector noise is taken into consideration.

We want to know when will the signal to noise go greater than $1$ if we continue to enlarge the summation range and neglect detector noise. The result is shown in next figure

\Spng{StoN}{Time Delay Signal to Noise with $L=1$}
Here we keep $l_{min}$ fixed at $1000$ and let $l_{max}$ vary up to $7000$, we can see that at $l_{max} \approx 4200$, minimum variance estimator $S/N \approx 1$. 
Also notice that there is no $\Theta B$, $EB$ or $BB$ polarization here, since if you look at the difference of ${}_{s} F_{lLl'}$ and ${}_{s} G_{lLl'}$, we would find that the noise for $\Theta B$, $EB$ and $BB$ estimator is much larger than $\Theta \Theta$, $\Theta E$ and $EE$ estimator. So we don't include those terms. 

The next thing we want to consider is the orthogonality of lensing and time delay quadratic estimator. Consider a lensing quadratic estimator given by Okamoto \& Hu
\bea
\hat{\phi}^{\alpha}_{LM} &=&\nonumber \frac{A_{L}^{\alpha}}{\sqrt{L(L+1)}}\sum_{l_{1}m_{1}}\sum_{l_{2}m_{2}} \\
&& (-1)^{M}\bigl(\begin{smallmatrix} l_1 & l_2 & L \\ m_1 & m_2 & -M  \end{smallmatrix}\bigr) h^{\alpha,\phi}_{l_{1}l_{2}}(L) a^{obs}_{l_{1}m_{1}}b^{obs}_{l_{2}m_{2}}
\eea
Here $h^{\alpha,\phi}_{l_{1}l_{2}}$ has a similar expression to $h^{\alpha,\phi}_{l_{1}l_{2}}$ with $g$ replaced by $f$. And we can compute the time delay contamination of this lensing estimator
\bea
&& \langle \hat{\phi}^{\alpha}_{LM} \rangle = \sqrt{L(L+1)} \phi_{LM} \nonumber \\
&& + \sqrt{L(L+1)} d_{LM} C^{\alpha}_{L}
\eea
where $C^{\alpha}_{L}$ is given by
\be
C^{\alpha}_{L} = \frac{\sum_{l_{1}l_{2}}h_{l_{1}l_{2}}^{\alpha,\phi}(L)g_{l_{1}Ll_{2}}^{\alpha}}{\sum_{l_{1}l_{2}}h_{l_{1}l_{2}}^{\alpha,\phi}(L)f_{l_{1}Ll_{2}}^{\alpha}}
\ee
Now we are mostly interested in $L\approx 30$ because that when lensing signal is at maximum. $C^{\Theta \Theta}_{30} \approx 0.00157$ so the contribution from time delay is actually negligible (for $\Theta \Theta $ case).
\bibliography{refs}
\end{document}

 \subsection{Comparison with Deflection estimator}
 
 It is interesting to compare this estimator with that obtained for the deflection angle at this stage in the calculation in \citet{Okamoto:2003zw}. Paraphrasing Eq (29) there leads to
 \be
 \hat\alpha_{LM} = \frac{A_L}{\sqrt{L(L+1}}\,\sum_{l_1,m_1,l_2m_2}
 (-1)^M \left(\begin{matrix} l_1& l_2 &L\cr m_1&m_2&-M \end{matrix}\right)
 g_{l_1l_2}
 a^{\rm obs}_{l_1m_1} a^{\rm obs}_{l_2m_2} 
 \ee
 where
 
 \be
 A_L \equiv L(L+1)(2L+1) \left[ \sum_{l_1l_2} g_{l_1l_2} f_{l_1Ll_2}\right]^{-1}
 \ee
