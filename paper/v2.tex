%\documentclass[prd,twocolumn,amsmath,amssymb,floatfix,superscriptaddress,nofootinbib,preprintnumbers]{revtex4-1}
%\documentclass[prl,amsmath,amssymb,floatfix,superscriptaddress,nofootinbib,preprintnumbers]{revtex4-1}
\documentclass[prl,amsmath,amssymb,floatfix,superscriptaddress,nofootinbib,twocolumn]{revtex4-1}

\def\be{\begin{equation}}
\def\ee{\end{equation}}
\def\bea{\begin{eqnarray}}
\def\eea{\end{eqnarray}}
\newcommand{\vs}{\nonumber\\}
\def\across{a^\times}
\def\tcross{T^\times}
\def\ccross{C^\times}
\newcommand{\ec}[1]{Eq.~(\ref{eq:#1})}
\newcommand{\eec}[2]{Eqs.~(\ref{eq:#1}) and (\ref{eq:#2})}
\newcommand{\Ec}[1]{(\ref{eq:#1})}
\newcommand{\eql}[1]{\label{eq:#1}}
\newcommand{\sfig}[2]{
\includegraphics[width=#2]{../plots/#1}
        }
\newcommand{\sfigr}[2]{
\includegraphics[angle=270,origin=c,width=#2]{#1}
        }
\newcommand{\sfigra}[2]{
\includegraphics[angle=90,origin=c,width=#2]{#1}
        }
\newcommand{\Sfig}[2]{
   \begin{figure}[thbp]
   \begin{center}
    \sfig{#1.pdf}{0.9\columnwidth}
    \caption{{\small #2}}
    \label{fig:#1}
     \end{center}
   \end{figure}
}
\newcommand\dirac{\delta_D}
\newcommand{\rf}[1]{\ref{fig:#1}}
\newcommand\rhoc{\rho_{\rm cr}}
\newcommand\zs{D_S}
\newcommand\dts{\Delta t_{\rm Sh}}
\newcommand\zle{D_L}
\newcommand\zsl{D_{SL}}


%USEFUL PACKAGES
\usepackage[utf8]{inputenc}
\usepackage{graphicx}
\usepackage{amssymb}
\usepackage{amsmath}
\usepackage{bm}
\usepackage{color}
\usepackage{enumitem}
\usepackage[linktocpage=true]{hyperref} 
\hypersetup{
    colorlinks=true,       % false: boxed links; true: colored links
    linkcolor=red,          % color of internal links
    citecolor=blue,        % color of links to bibliography 
    filecolor=magenta,      % color of file links
    urlcolor=blue           % color of external links
}
\usepackage[all]{hypcap} 

%USEFUL MACROS
%\usepackage{myterms}
\definecolor{darkgreen}{cmyk}{0.85,0.1,1.00,0} 

\newcommand{\SD}[1]{{\color{darkgreen} SD: #1}}
\newcommand{\WH}[1]{{\color{red} WH: #1}}
\newcommand{\wh}[1]{{\color{red} WH: #1}}
\newcommand{\AL}[1]{{\color{magenta} AL: #1}}
\newcommand{\MR}[1]{{\color{blue} MR: #1}}

\newcommand{\nl}{\\ \indent}
%

\begin{document}

\title{Distortions in the Surface of Last Scattering}


\author{\large Peikai Li}
\author{\large Scott Dodelson}
%\email{dodelson@fnal.gov}
\affiliation{Department of Physics, Carnegie Mellon University, Pittsburgh, Pennsylvania 15312, USA.}

\date{\today}

\begin{abstract}
The surface of last scattering of the photons in the cosmic microwave background is {\bf not} a spherical shell. Apart from its finite width, each photon experiences a difference gravitational potential along its journey to us, leading to different travel times in different directions. Since all photons were released at the same cosmic time, the photons with longer travel times started farther away from us than those with shorter times. Thus, the surface of last scattering is corrugated, a deformed spherical shell. We present a quadratic estimator that could provide a map of the time delays as a function of position on the sky. The signal to noise of this map could exceed unity on large scales.
\end{abstract}

\maketitle

\section{Distance to the Last Scattering Surface}

The theory of general relativity dictates that particles traveling through gravitational potential wells experience time delays~\cite{1964PhRvL..13..789S}. If two photons are emitted at the same time, then they will travel different distances depending upon the potential $\Phi$ through which they travel. In a cosmological context of an expanding universe, the difference in comoving distance $D$ is equal to  
\be
d(\hat n) = \frac{-2}{D} \int_0^D dD'\, \Phi\left(D'\hat n; t(D')\right)
\ee
where $t(D)$ is the age of the universe at the time that the photon is a distance $D$ from us. There is also a geometric time delay that is typically of the same size for a single lens, but when the path is through a series of peaks and troughs, the mean time delay is much smaller, of order $\theta_{RMS}$, so we neglect it here. 

Similarly~\cite{Hu:2001yq}, photons that comprise the cosmic microwave background (CMB) experience these same time delays or advances depending on the integrated potential through which they travel. Since photons do not decouple instantaneously from the electron-proton plasma, the surface of last scattering is often said to have a finite width, and a more accurate expression for the fractional difference in distance traveled is
\be
d^{\rm CMB}(\hat n) = \frac{-2}{D_*} \int_0^\infty dD e^{-\tau(D)}\, \Phi\left(D\hat n; t(D)\right)
\ee
where $\tau$ is the optical depth, which becomes very large at times smaller than $t_*$ or equivalently when $D>D_*$, the nominal distance to the last scattering surface. This directional-dependent change in the distance to last scattering is independent of its finite width and a phenomenon different than the angular deflections~\cite{Hu:2001tn,Lewis:2006fu} that have been captured by recent experiments~\cite{Smith:2007rg,Ade:2013tyw,Story:2014hni,Sherwin:2016tyf,Aghanim:2018oex}.

Although deflections and delays are two different phenomena, they share some similarities, especially in the case of the CMB. Both are determined by the integrated potential along the line of sight, although with slight different kernels, as depicted in Figure~\rf{kernel}. It is clear that they will be highly correlated, so as a first approximation, we might view the maps of the lensing potential created for example in \citet{Aghanim:2018oex} as maps of distance to the last scattering surface. Another similarity, one that has not yet been exploited, is that the formalism first proposed in \citet{Hu:2001tn} can be applied to the delays as well, and this is what we will do in this paper. We start though with the rather humbling calculation of \cite{Hu:2001yq} that the RMS fractional distance differences are a factor of ten smaller than the RMS angular deviations. Further, while the latter peaks at degree scales, the former peak on the largest scales.

\Sfig{kernel}{Kernel that weights the integral of the gravitational potential for the time delay examined here and the more carefully studied deflection angle.}

\section{Effect of distance changes on the CMB}
\newcommand\fd{d^{\rm CMB}}
\newcommand\tob{\Theta^{\rm obs}}
\newcommand\tu{\Theta}
\newcommand\td{\Theta^{\rm dist}}
A starting point is to assume that the last scattering surface is infinitely thin, note that the observed temperature in a given direction $\tob(\hat n)$ is the undistorted temperature $\tu(\hat n)$ plus a term proportional to the small fractional difference $\fd(\hat n)$:
\be
\tob(\hat n) = \tu(\hat n) + \frac{\partial\tu(\hat n)}{\partial \ln D_*}\, \fd(\hat n)
.\ee
The derivative can be evaluated by recalling that the undistorted temperature can be expanded in spherical harmonics with coefficients
\be
\tu_{lm} = (-1)^{l} \int \frac{d^{3}k}{(2\pi)^{3}}Y_{lm}^{*}(\hat{k})S(\vec k;\eta_{*})j_{l}(kD_{*}) 
\ee
where $S$ is the source function, dominated on small scales by the monopole and dipole, as in Eq. (22) of Ref.~\cite{Hu:2001bc}. The derivative then acts on the spherical Bessel function, and after expanding the fractional difference in spherical harmonics as well, we obtain
\be
\tob_{lm} = \tu_{lm} + \td_{lm}
\ee
with the first order term due to the distortion equal to
\be
\td_{lm} = \sum_{LM}\sum_{l'm'} \frac{\partial \tu_{LM}(D_{*})}{\partial \ln D_{*}} d_{l'm'} I_{lLl'}^{mMm'}
.\ee
Here, the integral over the product of three spherical harmonics is expressed as 
\be
I_{lLl'}^{mMm'}  = (-1)^m\,
\bigl(\begin{smallmatrix} l & L & l' \\ -m & M & m'  \end{smallmatrix}\bigr) F_{lLl'}
\ee
with
\be
F_{lLl'} \equiv \frac{\big[ (2l+1)(2L+1)(2l'+1)\big]^{1/2} }{\sqrt{4 \pi}} \bigl(\begin{smallmatrix} l & L & l' \\ 0 & 0 & 0  \end{smallmatrix}\bigr)
.\ee

We proceed as in Ref.~\cite{Hu:2001tn} by focusing on the expectation of off-diagonal (${l_1,m_1\ne l_2,m_2}$) terms quadratic in the observed moments:
\bea
\langle \tob_{l_1m_1} \tob_{l_2m_2} \rangle &&=
\sum_{LM} d_{LM} (-1)^M F_{l_2 l_1L}  \bigl(\begin{smallmatrix} l_1 & l_2 & L \\ m_1 & m_2 & -M  \end{smallmatrix}\bigr) \vs
&\times &\Big[ C_{l_1}^d +C_{l_2}^d\Big]
\vs\eea
where the spectrum
\be
C_{l}^d \equiv \int_0^\infty \frac{d^3k}{(2\pi)^3}\, P_S(k,\eta_*)\,j_l(kD_*)\,\frac{\partial j_l(kD_*)}{\partial\ln D_*}
\ee
with $P_S$ the power spectrum of the source. This expression is identical to the one for the undistorted CMB spectrum $C_l$ other than the replacement $j_l\rightarrow j_l'$. The two spectra are shown in Fig.~XXX. Therefore, following Ref.~\cite{Okamoto:2003zw}, we form the quadratic estimator for the fractional distance field
\be
\hat d_{LM} = A_{L} \sum_{l_1m_1l_2m_2} (-1)^M  \bigl(\begin{smallmatrix} l_1 & l_2 & L \\ m_1 & m_2 & -M  \end{smallmatrix}\bigr) g_{l_1l_2}(L)  \tob_{l_1m_1} \tob_{l_2m_2}
\ee
where
\be
g_{l_1l_2}(L)\equiv \frac{f_{l_1Ll_2}}{2C_{l_1}C_{l_2}}
\ee
and
\be
A_L \equiv (2L+1) \left\{ \sum_{l_1l_2} g_{l_1l_2}(L)f_{l_1Ll_2}\right\}^{-1}
.\ee 
This estimator has an expected value equal to the true $d_{LM}$ and a variance 
\be
\langle \hat d_{LM} \hat d^*_{L'M'}  \rangle = \delta_{LL'}\delta_{MM'} \left( C_L^{dd} + A_L \right)
\ee
with the first term on the right the signal and the second the noise. Fig.YYY shows the signal and noise as a function of $L$ for several experimental configurations.

 
\bibliography{refs}
\end{document}

 \subsection{Comparison with Deflection estimator}
 
 It is interesting to compare this estimator with that obtained for the deflection angle at this stage in the calculation in \citet{Okamoto:2003zw}. Paraphrasing Eq (29) there leads to
 \be
 \hat\alpha_{LM} = \frac{A_L}{\sqrt{L(L+1}}\,\sum_{l_1,m_1,l_2m_2}
 (-1)^M \left(\begin{matrix} l_1& l_2 &L\cr m_1&m_2&-M \end{matrix}\right)
 g_{l_1l_2}
 a^{\rm obs}_{l_1m_1} a^{\rm obs}_{l_2m_2} 
 \ee
 where
 
 \be
 A_L \equiv L(L+1)(2L+1) \left[ \sum_{l_1l_2} g_{l_1l_2} f_{l_1Ll_2}\right]^{-1}
 \ee

